\documentclass[a4paper, 12pt]{report}
\usepackage[utf8]{inputenc}
\usepackage{silence}
\usepackage{amsmath,amsfonts,amssymb,amsthm}
\usepackage{mathtools}
\usepackage{enumitem}
\usepackage{newunicodechar}
\usepackage{etoolbox}
\usepackage[margin=1.2in]{geometry}
\usepackage{stmaryrd}
\SetSymbolFont{stmry}{bold}{U}{stmry}{m}{n}
% https://tex.stackexchange.com/a/106719
\DeclareSymbolFont{sfletters}{OML}{cmbrm}{m}{it}
\DeclareMathSymbol{\slambda}{\mathord}{sfletters}{"15}
%https://tex.stackexchange.com/a/335857
\usepackage{microtype}
\usepackage[dvipsnames]{xcolor}
\usepackage{mathpartir}
\usepackage[style=alphabetic]{biblatex}
\usepackage{hyperref}
\usepackage{cleveref}
\usepackage{tikz}

\addbibresource{refs.bib}

\title{Normalization for algebraic effect handlers}
\author{Wiktor Kuchta}
\date{\vspace{-4ex}}

%\newcommand{\textbsf}[1]{\textbf{\textsf{#1}}}
%\newunicodechar{λ}{\mathsf{\lambda}}
%\newunicodechar{λ}{\mathbf{\lambda}}
%\newunicodechar{λ}{\boldsymbol{\lambda}}
%\newunicodechar{λ}{\lambda}
\newunicodechar{λ}{\slambda}
\newcommand{\keyword}[1]{\textsf{\textup{#1}}}
\newcommand{\Do}{\keyword{do}\;}
\newcommand{\Handle}{\keyword{handle}\;}
\newcommand{\Lift}[1]{\boldsymbol{[}#1\boldsymbol{]}}
\newcommand{\subst}[2]{\{#1/#2\}}
\newcommand{\E}{\mathcal{E}}
\newcommand{\K}{\mathcal{K}}
\renewcommand{\S}{\mathcal{S}}
\newcommand{\kT}{\mathsf{T}}
\newcommand{\kE}{\mathsf{E}}
\newcommand{\kR}{\mathsf{R}}
\newcommand{\Free}{\textrm{-}\mathrm{free}}
\newcommand{\Obs}{\mathrm{Obs}}
\newcommand{\N}{\mathbb{N}}
\newcommand{\Z}{\mathbb{Z}}
\newcommand{\norm}[1]{\left\lVert #1 \right\rVert}
\newcommand{\modulus}[1]{\left| #1 \right|}
\newcommand{\abs}{\modulus}
\newcommand{\ol}{\overline}
\DeclareMathOperator{\dom}{dom}
\newcommand{\+}{\enspace}
\newcommand{\gr}{\textcolor{ForestGreen}}

\newtheorem{corollary}{Corollary}
\newtheorem{lemma}{Lemma}
\newtheorem{theorem}{Theorem}

\newunicodechar{│}{\mid} % Digr vv
%\newunicodechar{╱}{/} % Digr FD
\newunicodechar{╱}{\mathbin{/}} % Digr FD
\newunicodechar{∷}{::} % Digr ::
\newunicodechar{□}{\square} % Digr OS
\newunicodechar{∅}{\emptyset} % Digr /0
\newunicodechar{∞}{\infty} % Digr 00
\newunicodechar{∂}{\partial} % Digr dP
\newunicodechar{α}{\alpha}
\newunicodechar{β}{\beta}
\newunicodechar{ξ}{\xi} % Digr c*
\newunicodechar{δ}{\delta} % Digr d*
\newunicodechar{ε}{\varepsilon}
\newunicodechar{φ}{\varphi}
\newunicodechar{γ}{\gamma} % Digr g*
\newunicodechar{θ}{\theta} % Digr h*
\newunicodechar{ι}{\iota} % Digr i*
\newunicodechar{κ}{\kappa}
\newunicodechar{μ}{\mu}
\newunicodechar{ν}{\nu}
\newunicodechar{π}{\pi}
\newunicodechar{ψ}{\psi}
\newunicodechar{ρ}{\rho}
\newunicodechar{σ}{\sigma}
\newunicodechar{τ}{\tau}
\newunicodechar{ω}{\omega}
\newunicodechar{η}{\eta} % Digr y*
\newunicodechar{ζ}{\zeta} % Digr z*
\newunicodechar{Δ}{\Delta}
\newunicodechar{Γ}{\Gamma}
\newunicodechar{Λ}{\Lambda}
\newunicodechar{Θ}{\Theta}
\newunicodechar{Φ}{\Phi} % Digr F*
\newunicodechar{Π}{\Pi}
\newunicodechar{Ψ}{\Psi} % digr Q*
\newunicodechar{Σ}{\Sigma} % digr S*
\newunicodechar{Ω}{\Omega} % digr W*
\newunicodechar{ℕ}{\N} % Digr NN 8469 nonstandard
\newunicodechar{ℤ}{\Z} % Digr ZZ 8484 nonstandard
\newunicodechar{∑}{\sum}
\newunicodechar{∏}{\prod}
\newunicodechar{∫}{\int}
\newunicodechar{∓}{\mp}
\newunicodechar{⌈}{\lceil} % Digr <7
\newunicodechar{⌉}{\rceil} % Digr >7
\newunicodechar{⌊}{\lfloor} % Digr 7<
\newunicodechar{⌋}{\rfloor} % Digr 7>
\newunicodechar{≅}{\cong} % Digr ?=
\newunicodechar{≡}{\equiv} % Digr 3=
\newunicodechar{◁}{\triangleleft} % Digr Tl
\newunicodechar{▷}{\triangleright} % Digr Tr
\newunicodechar{≤}{\le}
\newunicodechar{≥}{\ge}
\newunicodechar{≪}{\ll} % Digr <*
\newunicodechar{≫}{\gg} % Digr *>
\newunicodechar{≠}{\ne}
\newunicodechar{⊆}{\subseteq} % Digr (_
\newunicodechar{⊇}{\supseteq} % Digr _)
\newunicodechar{⊂}{\subset} % Digr (C
\newunicodechar{⊃}{\supset} % Digr C)
\newunicodechar{∩}{\cap} % Digr (U
\newunicodechar{∖}{\setminus} % Digr -\ 8726 nonstandard
\newunicodechar{∪}{\cup} % Digr )U
\newunicodechar{∼}{\sim} % Digr ?1
\newunicodechar{≈}{\approx} % Digr ?2
\newunicodechar{∈}{\in} % Digr (-
\newunicodechar{∋}{\ni} % Digr -)
\newunicodechar{∇}{\nabla} % Digr NB
\newunicodechar{∃}{\exists} % Digr TE
\newunicodechar{∀}{\forall} % Digr FA
\newunicodechar{∧}{\wedge} % Digr AN
\newunicodechar{∨}{\vee} % Digr OR
\newunicodechar{⊥}{\bot} % Digr -T
\newunicodechar{⊢}{\vdash} % Digr \- 8866 nonstandard
\newunicodechar{⊨}{\models} % Digr \= 8872 nonstandard
\newunicodechar{⊤}{\top} % Digr TO 8868 nonstandard
\newunicodechar{⇒}{\implies} % Digr =>
\newunicodechar{⊸}{\multimap} % Digr #> nonstandard
\newunicodechar{⇐}{\impliedby} % Digr <=
\newunicodechar{⇔}{\iff} % Digr ==
\newunicodechar{↔}{\leftrightarrow} % Digr <>
\newunicodechar{↦}{\mapsto} % Digr T> 8614 nonstandard
\newunicodechar{∘}{\circ} % Digr Ob
\newunicodechar{⊕}{\oplus} % Digr O+ 8853
\newunicodechar{⊗}{\otimes} % Digr OX 8855
\newunicodechar{⟦}{\llbracket} % Digr [[ 10214 nonstandard (needs pkg stmaryrd)
\newunicodechar{⟧}{\rrbracket} % Digr ]] 10215 nonstandard


% cursed
\WarningFilter{newunicodechar}{Redefining Unicode}
\newunicodechar{·}{\ifmmode\cdot\else\textperiodcentered\fi} % Digr .M
\newunicodechar{×}{\ifmmode\times\else\texttimes\fi} % Digr *X
\newunicodechar{→}{\ifmmode\rightarrow\else\textrightarrow\fi} % Digr ->
\newunicodechar{←}{\ifmmode\leftarrow\else\textleftarrow\fi} % Digr ->
\newunicodechar{⟨}{\ifmmode\langle\else\textlangle\fi} % Digr LA 10216 nonstandard
\newunicodechar{⟩}{\ifmmode\rangle\else\textrangle\fi} % Digr RA 10217 nonstandard
\newunicodechar{…}{\ifmmode\dots\else\textellipsis\fi} % Digr .,
\newunicodechar{±}{\ifmmode\pm\else\textpm\fi} % Digr +-
\newunicodechar{¬}{\ifmmode\lnot\else\textlnot\fi} % Digr NO

\begin{document}

\maketitle

\chapter{Introduction}
A programming language needs to be more than just a lambda calculus,
capable only of functional abstraction and evaluation of expressions.
Programs need to have an effect on the outside world and, thinking more locally,
in our programs we would like to have fragments that do not merely reduce to a value in isolation,
but also nontrivially affect the execution of surrounding code.
This is what we call computational effects, the typical examples of which are:
input/output, mutable state, exceptions, nondeterminism, and coroutines.

Today, we are at the mercy of programming language designers,
and we can only hope they include the
effects we would like to use and make sure that they all interact with
each other well.
A traditional way to remedy this is to (purely functionally) model effects using monads \cite{monads}.
This requires the use of monadic style, which among other things causes
stratification of code into two styles and does not always cooperate well with
the language's native facilities.
Moreover, monads do not compose in general, so modular programming with monads has some overheads.

In recent years a new promising approach has emerged: \textit{algebraic effects and handlers} \cite{Plotkin_2013}.
A language featuring algebraic effects lets programmers express all common computational effects as library code
and use them in the usual direct style.
This is done by decoupling effects into purely syntactic {\em operations} with their type signatures
and giving them meaning separately, here by using {\em effect handlers}.

Effect handlers may be thought of as resumable exceptions.
We can \textit{perform} an operation,
just as we can raise an exception in a typical high-level programming language.
The operation is then handled by the nearest enclosing \textit{handler} for the specific operation.
The handler receives the value given at perform-point and also,
unlike normal exception handlers,
a \textit{resumption} – a first-class functional value representing the rest
of the code to execute inside the handler from the perform-point.
By calling the resumption with a value we can resume at the perform-point,
as if performing the operation evaluated to the value.
But more interestingly, we can resume multiple times, or never, or store the resumption for later use.

To prevent crashes because of unhandled operations and aid understanding of effectful code,
languages with algebraic effects typically have powerful
\textit{type-and-effect systems}\footnote{
We will often say just ``type'' when referring to both types and effects.},
not only tracking the type of value an expression might evaluate to, but also
what kind of effects it may perform along the way.
Type-and-effect systems allow us to deduce some basic information about programs,
for example that a pure expression cannot evaluate to two different values,
or, in some systems, that it will {\em always} evaluate to a value.

In this work we demonstrate a (to our knowledge) novel proof
termination of evaluation (also known as normalization) for a
 language equipped with algebraic effects.
We obtain soundness as a corollary.
We also explain how the proof can carry over to other calculi.
Further, we explore the computational content of the proofs –
a normalization by evaluation algorithm for the calculus.

Termination of evaluation might be of interest for many reasons,
for example it simplifies formal semantics and reasoning.
It can also be considered as another safety property, just like ``never crashing''
(though in practice nonterminating programs are indistinguishable from very long-running ones).
Even though most programming languages support recursion, loops, and other
features allowing nontermination (eg. recursive types),
it might be useful to know that fragments that do not use these features cannot be blamed
for it.

Nontermination is sometimes also viewed as a computational effect,
which however does not neatly fit the algebraic effect framework.
Despite that, it can still be added as a built-in effect to be tracked by
the effect system, provided that we have sufficiently identified possible sources
of nontermination.
This is done, for example, in the Koka programming language \cite{koka}.

Normalization also implies that the calculus is sound as a logic, % TODO: what semantics? normal forms?
looking through the lens of the Curry-Howard isomorphism.
However, it is hard to give logical interpretations to effect annotations \cite{oleg}
and it is unclear how useful it is to treat calculi with algebraic effects as logics.

The primary tool for proving normalization (and many other properties) of % TODO to proving?
typed formal calculi is {\em logical relations}.
Logical relations are
type-indexed relations on terms that essentially help carry additional
information about {\em local} behavior of terms to make a type-derivation
directed proof of the {\em global} property (such as normalization)
have strong enough induction hypotheses and ultimately go through.

In this work we define logical relations\footnote{
	In the context of normalization, logical relations have many names, including:
	(Tait's) reducibility predicates, reducibility candidates, and saturated sets.
}
to prove termination
of evaluation in our language.
A novelty of our approach
is that the definition of the logical relations
is recursive, in other words it is a solution to a fixed-point equation.
%This is unlike any normalization proof we know of.
The fixed-point construction replaces the use of step-indexing % TODO cite
which appears in the logical relation construction we are most directly inspired by \cite{hwc}.

I would like to thank Dariusz Biernacki, Filip Sieczkowski, and Piotr Polesiuk
for their help with this work.

\chapter{The language}
% TODO name the language?

We will study the deep handler calculus and type-and-effect system formulated
in \cite{fscd19}.
% TODO: EM—DASH VS EN–DASH STYLE !!!
It is a refreshingly minimal language – the call-by-value lambda calculus with a few extensions
to be able to express the essence of algebraic effects.
There is only the universal operation, performed $\Do v$.
To be able to reach beyond the closest effect handler,
the \textit{lift} operator, written $\Lift{e}$, is introduced,
which makes operations in $e$ skip the nearest handler.
%When operations are performed inside the expression $e$ enclosed by lift,
%the nearest handler will be skipped and the operation will be handled by the next one instead.
%Naturally, the operator composes, so multiple enclosing lifts means multiple handlers skipped.
In contrast to most work on algebraic effects, the effect-tracking system here is structural –
we do not have any concept of predefined or user-defined signatures of effects.
Finally, the type system features polymorphic expressions and polymorphic operations.

The syntax of the calculus is shown in \cref{syntax}.
We use metavariables $x$ and $r$ to refer to expression variables.
As is standard, we work modulo $α$-equivalence and distinguish separate instances
of metavariables by using subscripts, superscripts, or adding primes.

Evaluation contexts are expressions with a hole ($□$) and
encode a left-to-right evaluation order.
We use the standard notation $K[x]$ for substituting $x$ for the hole in $K$.
There is a concept of {\em freeness} of evaluation contexts (\cref{freeness}),
a quantity increased by lifts and decreased by handlers.
Intuitively, an evaluation context is $n\Free$
if an operation performed inside the context (in place of the hole) % in the place?
would be handled by the $n$-th handler outside of the context.

Naturally, freeness is used in the operational semantics (\cref{reduction}),
where we can see that effect handling occurs if there
is a $0\Free$ context between the operation and the handler.
The resumption $v_c$ wraps the context $K$
with the same handler, which is why the effect handlers are called {\em deep}
– all operations are handled, not just the first one (as in so called {\em shallow} handlers).
The other rules are the standard $β$-reduction of the $λ$-calculus
and reductions in cases where an (effect-free) value is wrapped by an effect-related construct.
It is straightforward to see that reduction is deterministic
and compatible with respect to evaluation contexts.

In \cref{types} the type system is introduced.
We have three kinds: types, effects, and rows.
We use $α$ to range over type variables and
$Δ$ to range over type contexts, which assign kinds to type variables.
Similarly, contexts $Γ$ assign types to term-level variables.
We furthermore have the concept of {\em type-level substitutions}:
we write $Δ ⊢ δ ∷ Δ'$ if $δ$ maps type variables in $Δ'$ to types of the corresponding kinds,
well-formed in $Δ$.
We assume that all types are well-formed (well-kinded) in the appropriate contexts in the sense of \cref{kinding}.


Type-level substitutions are used only in the rule for $\Do\!\!\!$
to instantiate the polymorphic variables $Δ'$ of effect $Δ'.\,τ_1 \Rightarrow τ_2$.
Complementarily, the rule for $\Handle\!\!$ states that the effect handler clause $e_h$
has to be essentially polymorphic in $Δ'$.

We can intuitively understand effect rows as follows:
 if $e : τ ╱ ρ$ and effect $ε$ appears at position $n$ in $ρ$, then
subexpressions performing $ε$ are wrapped by $n$ lifts inside $e$.
Clearly, the order of effects in the row matters.
The type system enables {\em row polymorphism}, % TODO: CITE
since we can have a row ending in a polymorphic type variable.
Such rows are usually called {\em open} (and {\em closed} otherwise). %, while monomorphic rows are called {\em closed}.

In \cref{subtyping} the subtyping rules are introduced.
Their main purpose is to allow
effect rows to be subsumed by rows with more effects at the end.
For that to be of use in more places,
we also have subsumption rules for polymorphic types and function types. % polymorphic?
In particular, closed rows are subsumed by open rows. % TODO: trivial forall on the left

% ORDER! polymorphism

\begin{figure}
\begin{align*}
	%\textrm{Var} ∋ f,r,x,y,…    &                    \\%&\textrm{(variables)}\\
	v,u          &::= x │ λx.\,e &\textrm{(values)}\\
	e            &::=
		v │ e\;e │ \Lift{e} │ \Do v │ \Handle e\,\{x,r.\,e;\,x.\,e\}
		&\textrm{(expressions)}\\
	K            &::=
		□ │ K\;e │ v\;K │ \Lift{K} │ \Handle K\,\{x,r.\,e;\,x.\,e\}
		&\textrm{(evaluation contexts)}
\end{align*}
\caption{Syntax.}
\label{syntax}
\end{figure}

\begin{figure}
\begin{mathpar}
	\inferrule
		{ }
		{0\Free(□)}

	\inferrule
		{n\Free(K)}
		{n\Free(K\;e)}

	\inferrule
		{n\Free(K)}
		{n\Free(v\;K)}

	\inferrule
		{n\Free(K)}
		{{n+1}\Free(\Lift{K})}

	\inferrule
		{{n+1}\Free(K)}
		{n\Free(\Handle K\,\{x,r.\,e_h;\,x.\,e_r\})}
\end{mathpar}
\caption{Evaluation context freeness.}
\label{freeness}
\end{figure}

\begin{figure}
\begin{mathpar}
	\inferrule
		{e_1 ↦ e_2}
		{K[e_1] → K[e_2]}

	(λx.\,e)\;v ↦ e\subst{v}{x} % SUBST

	\Lift{v} ↦ v

	\inferrule
		{0\Free(K) \\ v_c = λz.\,\Handle K[z]\,\{x,r.\,e_h;\,x.\,e_r\}}
		{\Handle K[\Do v]\,\{x, r.\,e_h;\,x.\,e_r\} ↦ e_h\subst{v}{x}\subst{v_c}{r}} % SUBST

	\Handle v\,\{x,r.\,e_h;\,x.\,e_r\} ↦ e_r\subst{v}{x} % SUBST
\end{mathpar}
\caption{Single-step reduction.} % TODO "Contraction"???
\label{reduction}
\end{figure}

\clearpage

\begin{figure}
\begin{mathpar}
	κ ::= \kT │ \kE │ \kR

	%(type variables): α,β,…

	%Δ ::= · │ α::κ, Δ

	%Γ ::= · │ x:τ, Γ

	σ,τ,ε,ρ ::=
		α │ τ→_ρτ │ ∀α::κ.\,τ │ ι │ Δ.\,τ \Rightarrow τ │ ε · ρ \\

	\inferrule
		{x : τ ∈ Γ}
		{Δ;Γ ⊢ x : τ ╱ ι}

	\inferrule
		{Δ ⊢ τ_1 ∷ \kT \\ Δ;Γ, x:τ_1 ⊢ e : τ_2 ╱ ρ}
		{Δ;Γ ⊢ λx.\,e : τ_1 →_ρ τ_2 ╱ ι}

	\inferrule
		{Δ;Γ ⊢ e_1 : τ_1 →_ρ τ_2 ╱ ρ \\ Δ;Γ ⊢ e_2 : τ_1 ╱ ρ}
		{Δ;Γ ⊢ e_1\,e_2 : τ_2 ╱ ρ}

	\inferrule
		{Δ ⊢ ε ∷ \kE \\ Δ;Γ ⊢ e : τ ╱ ρ}
		{Δ;Γ ⊢ \Lift{e} : τ ╱ ε·ρ}

	\inferrule
		{Δ,α∷κ;Γ ⊢ e : τ ╱ ι}
		{Δ;Γ ⊢ e : ∀α∷κ.\,τ ╱ ι}

	\inferrule
		{Δ ⊢ σ ∷ κ \\ Δ;Γ ⊢ e : ∀α∷κ.\, τ ╱ ρ}
		{Δ;Γ ⊢ e : τ\subst{σ}{α} ╱ ρ}

	\inferrule
		{Δ ⊢ τ_1 <: τ_2 \\ Δ ⊢ ρ_1 <: ρ_2 \\ Δ;Γ ⊢ e : τ_1 ╱ ρ_1}
		{Δ;Γ ⊢ e : τ_2 ╱ ρ_2}

	\inferrule
		{Δ;Γ ⊢ v : δ(τ_1) ╱ ι \\ Δ ⊢ δ ∷ Δ' \\ Δ ⊢ Δ'.\, τ_1 \Rightarrow τ_2 ∷ \kE}
		{Δ;Γ ⊢ \Do v : δ(τ_2) ╱ (Δ'.\, τ_1 \Rightarrow τ_2)}

	\inferrule
		{Δ;Γ ⊢ e : τ ╱ (Δ'.\,τ_1 \Rightarrow τ_2) · ρ \\
		Δ,Δ';Γ,x:τ_1,r:τ_2→_ρ τ_r ⊢ e_h : τ_r ╱ ρ \\
		Δ;Γ, x:τ ⊢ e_r : τ_r ╱ ρ}
		{Δ;Γ ⊢ \Handle e\;\{x,r.\,e_h;x.\,e_r\} : τ_r ╱ ρ}

\end{mathpar}
\caption{Type system.}
\label{types}
\end{figure}

\begin{figure}
\begin{mathpar}
	\inferrule
		{α ∷ κ ∈ Δ}
		{Δ ⊢ α ∷ κ}

	\inferrule
		{Δ ⊢ τ_1 ∷ \kT \\ Δ ⊢ ρ ∷ \kR \\ Δ ⊢ τ_2 ∷ \kT}
		{Δ ⊢ τ_1 →_ρ τ_2 ∷ \kT}

	\inferrule
		{Δ, α :: κ ⊢ τ ∷ \kT}
		{Δ ⊢ ∀α::κ.\, τ ∷ \kT}

	\inferrule
		{ }
		{Δ ⊢ ι ∷ \kR}

	\inferrule
		{Δ ⊢ ε ∷ \kE \\ Δ ⊢ ρ ∷ \kR}
		{Δ ⊢ ε · ρ ∷ \kR}

	\inferrule
		{Δ,Δ' ⊢ τ_1 ∷ \kT \\ Δ,Δ' ⊢ τ_2 ∷ \kT}
		{Δ ⊢ Δ'.\, τ_1 \Rightarrow τ_2 ∷ \kE}

	%Δ ⊢ δ ∷ Δ' ⇔ \dom(δ) = \dom(Δ') ∧ ∀α∈\dom(δ).\, Δ ⊢ δ(α) ∷ Δ'(α)
\end{mathpar}
%\caption{Well-formedness of types, rows, and type substitution.}
\caption{Well-formedness of types and rows.}
\label{kinding}
%TODO: inductive definition of well-formed type substitutions?
%TODO: syntax of Γ and Δ
\end{figure}
\begin{figure}
\begin{mathpar}
	\inferrule
		{ }
		{Δ ⊢ σ <: σ}
		%\textsc{(S-Refl)}

	\inferrule
		{Δ ⊢ ρ ∷ \kR}
		{Δ ⊢ ι <: ρ}
		%\textsc{(S-Empty)}

	\inferrule
		{Δ ⊢ ρ_1 <: ρ_2}
		{Δ ⊢ ε·ρ_1 <: ε·ρ_2}
		%\textsc{(S-Extend)}

	\inferrule
		{Δ ⊢ τ_2^1 <: τ_1^1 \\ Δ ⊢ ρ_1 <: ρ_2 \\ Δ ⊢ τ_1^2 <: τ_2^2}
		{Δ ⊢ τ_1^1 →_{ρ_1} τ_1^2 <: τ_2^1 →_{ρ_2} τ_2^2}
		%\textsc{(S-Arrow)}

	\inferrule
		{Δ, α ∷ κ ⊢ τ_1 <: τ_2}
		{Δ ⊢ ∀α∷κ.\, τ_1 <: ∀α∷κ. \,τ_2}
		%\textsc{(S-ForAll)}
\end{mathpar}
\caption{Subtyping.}
\label{subtyping}
\end{figure}

\chapter{The logical relation}
The logical relation is inspired by \cite{hwc}.
Some changes are due to language differences:
we have only one universal operation which simplifies the treatment of effects,
polymorphism does not manifest at the expression level (we do not have type lambdas),
and our operations can be polymorphic.
Instead of a binary step-indexed relation,
our goal is to build a unary relation without step-indexing.

\section{Definition}

We begin with defining the interpretations of kinds.
We call them the spaces of \textit{semantic types} or,
in the specific cases of $\kE$ and $\kR$, \textit{semantic effects}:
\begin{align*}
	⟦\kT⟧ &= \mathcal{P}(\textrm{CVal}) = \mathsf{Type}\\
	⟦\kE⟧ &= \mathcal{P}(\textrm{CExp}×\{0\}×\mathsf{Type}) \\
	⟦\kR⟧ &= \mathcal{P}(\textrm{CExp}×ℕ×\mathsf{Type})
\end{align*}
%TODO semantic types/effects are overapproximations of possible behaviors? or sth
We write $\textrm{CVal}$ and $\textrm{CExp}$ for the sets of closed values and closed expressions respectively, so
elements of $\mathsf{Type}$ are simply sets of closed values.
Semantic effects are sets of triples that aim to describe a situation
in which an expression being evaluated performs an effect.
The components of such a triple are:
the operation with its argument,
the freeness of the enclosing context beyond which the operation can be handled,
and the semantic type of values we can call the resumption with.

We interpret type contexts as mappings from type variables to semantic types:
$$⟦Δ⟧ = \{ η │ \dom(η) = \dom(Δ) ∧ ∀α∷κ∈Δ.\,η(α) ∈ ⟦κ⟧ \}.$$

We define the interpretations of types and effects as well as
relations $\E$ on expressions and $\S$ on control-stuck terms by structural induction.
We parameterize the definitions by a mapping $η$ from type variables to
semantic types.

The interpretations of types are mostly standard,
but we additionally have to consider effect annotations,
as we can see in the case of the function type.

A polymorphic type is interpreted, as usual,
as the intersection of the interpretations
for all possible choices of the semantic type for the type variable.
Interestingly, we can see that a polymorphic effect is interpreted as the {\em union} of the interpretations
of the effect for all possible choices of the semantic types for the type variables.
\begin{align*}
	⟦α⟧_η &= η(α) \\
	%\\
	⟦τ_1 →_ρ τ_2⟧_η
	 &= \{ λx.\,e \mid ∀v∈⟦τ_1⟧_η.\, e\subst{v}{x} ∈ \E⟦τ_2/ρ⟧_η \} \\
	⟦∀α::κ.\,τ⟧_η
	&= \{ v \mid ∀μ∈⟦κ⟧.\, v ∈ ⟦τ⟧_{[α↦μ]η} \} \\
	%\\
	⟦Δ.\,τ_1 \Rightarrow τ_2⟧_η &= \{(\Do v,0,⟦τ_2⟧_{ηη'}) \mid η'∈⟦Δ⟧ ∧ v∈⟦τ_1⟧_{ηη'} \} \\
	⟦ε · ρ⟧_η &= ⟦ε⟧_η ∪ \{(e, n+1, μ) \mid (e, n, μ)∈⟦ρ⟧_η \} \\
	⟦ι⟧_η &= ∅
\end{align*}
\begin{align*}
	\E⟦τ/ρ⟧_η &=
	\{ e \mid ∃v∈⟦τ⟧_η.\, e →^* v ∨ ∃e'∈\S⟦τ/ρ⟧_η.\, e →^* e' \} \\
	\S⟦τ/ρ⟧_η &= \{ K[e] \mid ∃n,μ.\, (e,n,μ)∈⟦ρ⟧_η  ∧ n\Free(K) ∧ ∀u∈μ.\, K[u]∈ \E⟦τ/ρ⟧_η \}
\end{align*} \indent
The definition of the set $\E⟦τ/ρ⟧_η$ is recursive and we interpret it inductively.
Note the similarities to the interpretation of function types in the definition of
$\S⟦τ/ρ⟧_η$.
The intuition is that an element of $\E⟦τ/ρ⟧_η$
evaluates to a value of type $τ$ while possibly performing an effect in $ρ$ and
getting control-stuck
finitely many times along the way.
When the term is control-stuck, we do not have a handler,
but we do know the return type of the operation,
so we resume with all possible values in parallel.

This is a very local view of effectful computations,
perhaps giving the false impression that all handlers immediately resume with a value and
do nothing more (this is called the {\em reader} effect and is equivalent to dynamically-scoped variables).
In reality, much can happen between
performing the operation and it finally returning.
Nevertheless, this local view suffices for our purposes and
accurately reflects how much of the computation remains in the scope of a handler
throughout evaluation.
% in parallel but finitely many times? ;/

\begin{figure}
\begin{tikzpicture}

	\node (e) at (0,0.5) {$\textit{times}\;(\Do ())\,\textit{tick}$};
	\node (e0) at (0,-1) {$\textit{times}\;0\,\textit{tick}$};
	\node (e1) at (4,-1) {$\textit{times}\;1\,\textit{tick}$};
	\node (e2) at (8,-1) {$\textit{times}\;2\,\textit{tick}$};
	\node (einf) at (12,-1) {$\hdots$};
	\draw[dashed,->] (e.south) -- (e0.north);
	\draw[dashed,->] (e.south) -- (e1.north);
	\draw[dashed,->] (e.south) -- (e2.north);
	\draw[dashed,->] (e.south) -- ([yshift=4] einf.north);

	\node (e0a) at (0,-2.5) {$()$};
	\node (e1a) at (4,-2.5) {$\Lift{\Do ()}; \textit{times}\;0\,\textit{tick}$};
	\node (e2a) at (8,-2.5) {$\Lift{\Do ()}; \textit{times}\;1\,\textit{tick}$};
	%\node (einfa) at (12,-1.85) {$\ddots$};

	\draw[->] (e0.south) -- (e0a.north) node[very near end,right=-3]{\textsubscript{*}};
	\draw[->] (e1.south) -- (e1a.north) node[very near end,right=-3]{\textsubscript{*}};
	\draw[->] (e2.south) -- (e2a.north) node[very near end,right=-3]{\textsubscript{*}};

	\node (e1b) at (4,-4) {$\Lift{()}; \textit{times}\;0\,\textit{tick}$};
	\node (e2b) at (8,-4) {$\Lift{()}; \textit{times}\;1\,\textit{tick}$};

	\draw[dashed,->] (e1a.south) -- (e1b.north);
	\draw[dashed,->] (e2a.south) -- (e2b.north);

	\node (e1c) at (4,-5.5) {$()$};
	\node (e2c) at (8,-5.5) {$\Lift{\Do ()}; \textit{times}\;0\,\textit{tick}$};

	\draw[->] (e1b.south) -- (e1c.north) node[very near end,right=-3]{\textsubscript{*}};
	\draw[->] (e2b.south) -- (e2c.north) node[very near end,right=-3]{\textsubscript{*}};

	\node (e2d) at (8,-7) {$\Lift{()}; \textit{times}\;0\,\textit{tick}$};

	\draw[dashed,->] (e2c.south) -- (e2d.north);

	\node (e2e) at (8,-8.5) {$()$};

	\draw[->] (e2d.south) -- (e2e.north) node[very near end,right=-3]{\textsubscript{*}};

	\node (einfa) at (12,-5.5) {$\vdots$};
	\node (einfb) at (12,-10) {$\ddots$};

\end{tikzpicture}
\caption{
	An illustration of the reduction trace for $\textit{times}\;(\Do ())\,\textit{tick} ∈
	\E⟦\textsf{unit}/ \textsf{askNat} · \textsf{tick} · ι⟧$,
	where $\textsf{askNat} = \textsf{unit}\Rightarrow\textsf{nat}$
	and $\textsf{tick} = \textsf{unit}\Rightarrow\textsf{unit}$.
	For clarity, we assume the unit type, the inductive naturals, and a function \textit{times} for invoking a thunk $n$ times,
	but these can be realized using Church encodings.
	We set $\textit{tick} = λx.\,\Lift{\Do ()}$.
	Here, the \textsf{tick} effect does not seem to do much, but
	a handler for it could, for instance, increment an external counter.
	The semicolon is standard syntax sugar:
	$e_1;e_2$ is expanded to $(λx.\,e_2)\;e_1$ with $x$ not free in $e_2$.
	The normal arrow is the usual reduction, while the dashed one represents substituting
	a value from the operation return type in a stuck term
	in $\S⟦\textsf{unit}/ \textsf{askNat} · \textsf{tick} · ι⟧$.
	This tree is well-founded despite having infinite depth.
}
\end{figure}

More graphically,
we can associate with an element in $\E⟦τ/ρ⟧_η$ what we will call a {\em reduction trace}
% TODO: quotes?
– a tree where nodes are expressions in $\E⟦τ/ρ⟧_η$,
leaves are values in $⟦τ⟧_η$, and edges are either (maximal) multi-step reductions
or branches from $\S⟦τ/ρ⟧_η$ for each element of $μ$,
which substitute the value for the operation.%
\footnote{
	This is also how we could view the structure of values of the inductive type $\E⟦τ/ρ⟧_η$,
	had we formalized the definitions in a (dependent) type theory.
}
Note that different choices of values to resume with could
theoretically lead to getting stuck a different number of times,
so the depth of the tree is not uniform, and it could even be infinite.
Still, these trees are well-founded thanks to the choice of the inductive interpretation.
\begin{align*}
	\E⟦τ/ρ⟧_η(X) &=
	\{ e \mid ∃v∈⟦τ⟧_η.\, e →^* v ∨ ∃e'∈\S⟦τ/ρ⟧_η(X).\, e →^* e' \} \\
	\S⟦τ/ρ⟧_η(X) &= \{ K[e] \mid ∃n,μ.\, (e,n,μ)∈⟦ρ⟧_η  ∧ n\Free(K) ∧ ∀u∈μ.\, K[u]∈X \} \\
	\E⟦τ/ρ⟧_η &= \bigcap \{ X │ \E⟦τ/ρ⟧_η(X) ⊆ X \} \\
	\S⟦τ/ρ⟧_η &= \S⟦τ/ρ⟧_η(\E⟦τ/ρ⟧_η)
\end{align*} \indent
To describe the construction in more detail, we temporarily overload notation and define
operators $\S⟦τ/ρ⟧_η$ and $\E⟦τ/ρ⟧_η$ on sets of expressions (denoted by $X$).
They are clearly monotone,
so by the Knaster-Tarski theorem % citation
the fixed-point equation $\E⟦τ/ρ⟧_η(X) = X$ has a least solution.\footnote{
	The operators are not Scott-continuous because of potentially infinite semantic
	types $μ$.
	This is why the construction from the Kleene fixed-point theorem (union of
	iterations of the operator on the empty set)
	would fail to be a solution.
	With this definition the reduction traces have bounded depth.
	The application compatibility lemma would fail:
	it grafts different reduction traces to the leaves of a
	potentially infinitely branching reduction trace,
	so the bounded depth is not preserved.
}
Moreover, it can be characterized as the intersection of all
$\E⟦τ/ρ⟧_η$-closed sets.
We immediately obtain the following principle:

\begin{lemma}[Tarski induction principle]\label{tarski-induction}
	If $\E⟦τ/ρ⟧_η(X) ⊆ X$, then $\E⟦τ/ρ⟧_η ⊆ X$.
\end{lemma}

By expanding out the definition of the function $\E⟦τ/ρ⟧_η$
and treating $X$ as a predicate $P$,
we get the more familiar principle of structural induction on $\E⟦τ/ρ⟧_η$.
\begin{lemma}[Induction principle]\label{induction}
	Assume $P$ is a predicate on closed expressions and
\begin{itemize}
	\item if $e$ evaluates to a value in $⟦τ⟧_η$, then $P(e)$ holds; and
	\item if $e$ reduces to some $K[e']$ such that there exist $(e',n,μ)∈⟦ρ⟧_η$ such that $n\Free(K)$
		and $P(K[u])$ holds for all $u∈μ$, then $P(e)$ holds.
\end{itemize}
	Then $P(e)$ holds for all $e ∈ \E⟦τ/ρ⟧_η$.
\end{lemma}

We also note some straightforward but useful properties of the relations.
\begin{lemma}[Value inclusion]\label{value-inclusion}
	For any $τ$ and $ρ$ we have $⟦τ⟧_η ⊆ \E⟦τ/ρ⟧_η$.
\end{lemma}

\begin{lemma}[Control-stuck inclusion]\label{stuck-inclusion}
For any $τ$ and $ρ$ we have $\S⟦τ/ρ⟧_η ⊆ \E⟦τ/ρ⟧_η$.
\end{lemma}

\begin{lemma}[Closedness under antireduction]\label{antireduction}
If $e →^* e' ∈ \E⟦τ/ρ⟧_η$, then $e∈\E⟦τ/ρ⟧_η$.
\end{lemma}

\begin{lemma}[Weakening]\label{weakening}
	If $η'$ extends $η$,
	then $\E⟦τ/ρ⟧_η = \E⟦τ/ρ⟧_{η'}$.
\end{lemma}

\begin{lemma}[Monotonicity in types]\label{mono}
	If $⟦τ_1⟧_η ⊆ ⟦τ_2⟧_η$ and $⟦ρ_1⟧_η ⊆ ⟦ρ_2⟧_η$,
	then $\S⟦τ_1/ρ_1⟧_η ⊆ \S⟦τ_2/ρ_2⟧_η$
	and $\E⟦τ_1/ρ_1⟧_η ⊆ \E⟦τ_2/ρ_2⟧_η$.
\end{lemma}

\section{Compatibility lemmas}
We want to establish that $⊢ e : τ ╱ ι$ implies $e ∈ \E⟦τ/ι⟧$.
For this purpose we will prove a semantic counterpart of each typing rule.
First, we need to define a counterpart to the typing judgment.
Unlike typing judgments, our relations are on closed terms only,
so we get around that by using substitution.
We define semantic entailment as follows:
$$Δ;Γ ⊨ e : τ ╱ ρ ⇔ ∀η∈⟦Δ⟧.\, ∀γ∈⟦Γ⟧_η.\,γ(e) ∈ \E⟦τ/ρ⟧_η,$$
where $⟦Γ⟧_η = \{ γ │ \dom(γ) = \dom(Γ) ∧ ∀x:τ∈Γ.\,γ(x) ∈ ⟦τ⟧_η\}$ contains expression-level
variable substitutions.

\begin{lemma}[Variable compatibility]
	$$
	\inferrule
		{x : τ ∈ Γ}
		{Δ;Γ ⊨ x : τ ╱ ι}
	$$
\end{lemma}
\begin{proof}
Assume $x : τ ∈ Γ$.
We want to prove $Δ;Γ ⊨ x : τ ╱ ι$.
Take any $η∈⟦Δ⟧$ and $γ∈⟦Γ⟧_η$.
We want to show $γ(x) ∈ \E⟦τ/ι⟧_η$.
From the definition of $⟦Γ⟧_η$
we know that $γ(x) ∈ ⟦τ⟧_η$,
so by \cref{value-inclusion} we have $γ(x) ∈ \E⟦τ/ι⟧_η$.
\end{proof}

\begin{lemma}[Abstraction compatibility]
	$$
	\inferrule
		{Δ ⊢ τ_1 ∷ \kT \\ Δ;Γ, x:τ_1 ⊨ e : τ_2 ╱ ρ}
		{Δ;Γ ⊨ λx.\,e : τ_1 →_ρ τ_2 ╱ ι}
	$$
\end{lemma}
\begin{proof}
Assume $Δ ⊢ {τ_1 ∷ \kT}$ and $Δ;Γ,{x:τ_1} ⊨ e : τ_2 ╱ ρ$.
% Note: kind assumption necessary so that τ_1 →_ρ τ_2 well formed
We want to prove $Δ;Γ ⊨ λx.\,e : {τ_1 →_ρ τ_2} ╱ ι$.
Take any $η∈⟦Δ⟧$ and $γ∈⟦Γ⟧_η$.
By \cref{value-inclusion} it suffices to show
$γ(λx.\,e) = λx.\,γ(e) ∈ ⟦τ_1 →_ρ τ_2⟧_η$.
So take any $v ∈ ⟦τ_1⟧_η.$
We need to show $γ(e)\subst{v}{x} ∈ \E⟦τ_2/ρ⟧_η$.
Let $γ' = γ'[x↦v]$.
Then $γ' ∈ ⟦Γ,x:τ_1⟧_η$, so $γ(e)\subst{v}{x} = γ'(e) ∈ \E⟦τ_2/ρ⟧_η.$
\end{proof}

For clarity of presentation,
in the following we will assume $Γ$ empty.
The lemmas in full generality can then be proven simply by
substituting an interpretation of $Γ$.

\begin{lemma}[Lift compatibility]
	$$
	\inferrule
		{Δ ⊢ ε ∷ \kE \\ Δ; ⊨ e : τ ╱ ρ}
		{Δ; ⊨ \Lift{e} : τ ╱ ε·ρ}
	$$
\end{lemma}
\begin{proof}
Assume $Δ ⊢ τ ∷ \kT$, $Δ ⊢ ε ∷ \kE$, and $Δ ⊢ ρ ∷ \kR$.
Take any $η∈⟦Δ⟧$.
We will show by induction on $e∈\E⟦τ/ρ⟧_η$ that $\Lift{e} ∈ \E⟦τ/ε·ρ⟧_η$.

If $e →^* K[e']$ and there exists
$(e', n, μ) ∈ ⟦ρ⟧_η$ such that $n\Free(K)$ and
for all $u∈μ$ the induction hypothesis holds for $K[u]$,
then we have $(e', n+1, μ) ∈ ⟦ε·ρ⟧_η$, $n+1\Free(\Lift{K})$,
and $∀u∈μ.\, \Lift{K[u]} ∈ \E⟦τ/ε·ρ⟧_η$.
So $\Lift{K[e']} ∈ \S⟦τ/ε·ρ⟧_η$ and $\Lift{e} ∈ \E⟦τ/ε·ρ⟧_η$ by antireduction.

If $e →^* v ∈ ⟦τ⟧_η$, then
$\Lift{e} →^* \Lift{v} → v$,
so $\Lift{e} ∈ \E⟦τ/ε·ρ⟧_η$.
\end{proof}

\begin{lemma}[Application compatibility]
	$$
	\inferrule
		{Δ; ⊨ e_1 : τ_1 →_ρ τ_2 ╱ ρ \\ Δ; ⊨ e_2 : τ_1 ╱ ρ}
		{Δ; ⊨ e_1\,e_2 : τ_2 ╱ ρ}
	$$
\end{lemma}
\begin{proof}
%Assume $Δ;Γ ⊨ e_1 : τ_1 →_ρ τ_2 ╱ ρ$ and $Δ;Γ ⊨ e_2 : τ_1 ╱ ρ$.
Fix any well-formed $Δ$ and $τ_1 →_ρ τ_2$.
Take any $η∈⟦Δ⟧$ and $e_2 ∈ \E⟦τ_1/ρ⟧_η$.
We will show by induction on $e_1∈\E⟦τ_1→_ρ τ_2/ρ⟧_η$ that
$e_1\;e_2 ∈ \E⟦τ_2/ρ⟧_η$.

If $e_1 →^* K_1[e_1']$ and
there exists $(e_1',n,μ)∈⟦ρ⟧_η$ such that $n\Free(K_1)$ and for all $u∈μ$
the inductive hypothesis holds for $K_1[u]$,
then $K_1[e_1']\;e_2 ∈ \S⟦τ_2/ρ⟧_η$, since $n\Free(K_1\;e_2)$.
By antireduction $e_1\;e_2 ∈ \E⟦τ_2/ρ⟧_η$.

Now assume $e_1 →^* (λx.\,e) ∈ ⟦τ_1 →_ρ τ_2/ρ⟧_η$.
We will show by induction on $e_2 ∈ \E⟦τ_1/ρ⟧_η$ that $(λx.\,e)\;e_2 ∈ \E⟦t_2/ρ⟧_η$
and the claim will follow by antireduction.

If $e_2 →^* K_2[e_2']$ and
there exists $(e_2',n,μ)∈⟦ρ⟧_η$ such that $n\Free(K_2)$ and for all $u∈μ$
the inductive hypothesis holds for $K_2[u]$,
then $(λx.\,e)\;K_2[e_2'] ∈ \S⟦τ_2/ρ⟧_η$, since $n\Free((λx.\,e)\;K_2)$.
By antireduction $(λx.\,e)\;e_2 ∈ \E⟦τ_2/ρ⟧_η$.

If $e_2 →^* v ∈ ⟦τ_1⟧_η$, then
$(λx.\,e)\;e_2 →^* (λx.\,e)\;v → e\subst{v}{x} ∈ \E⟦τ_2/ρ⟧_η$.
\end{proof}

\begin{lemma}[Handle compatibility]
	$$
	\inferrule
		{Δ; ⊨ e : τ ╱ (Δ'.\,τ_1 \Rightarrow τ_2) · ρ \\
		Δ,Δ';x:τ_1,r:τ_2→_ρ τ_r ⊨ e_h : τ_r ╱ ρ \\
		Δ; x:τ ⊨ e_r : τ_r ╱ ρ}
		{Δ; ⊨ \Handle e\;\{x,r.\,e_h;x.\,e_r\} : τ_r ╱ ρ}
	$$
\end{lemma}
\begin{proof}
Assume $Δ, Δ' ; x:τ_1, r:τ_2 →_ρ τ_r ⊨ e_h : τ_r ╱ ρ$ and
$Δ; x:τ ⊨ e_r : τ_r ╱ ρ$.
%We want to show
%$Δ;Γ ⊨ \Handle e\,\{x,r.\,e_h;\;x.\,e_r\} : τ_r ╱ ρ$.
Let $h$ stand for $\{x,r.\,e_h;\,x.\,e_r\}$.
Take any $η∈⟦Δ⟧$.
We will show by induction on $e∈\E⟦τ/(Δ'.\,τ_1 \Rightarrow τ_2) · ρ⟧_η$
that $\Handle e\,h ∈ \E⟦τ_r/ρ⟧_η$.
Note that only $τ_1$ and $τ_2$ require $Δ'$ to be in context.

If $e →^* v ∈ ⟦τ⟧_η$,
then $\Handle e\,h →^* e_r\subst{v}{x} ∈ \E⟦τ_r/ρ⟧_η$,
so the claim follows by antireduction.

Now assume $e →^* K[e']$ and
we have $(e',n,μ) ∈ ⟦(Δ'.\,τ_1 \Rightarrow τ_2) · ρ⟧_η$
such that $n\Free(K)$ and
for all $u∈μ$ the induction hypothesis holds for $K[u]$.

If $n=0$, then $(e',n,μ) ∈ ⟦τ_1 \Rightarrow τ_2⟧_{ηη'}$ for some $η'∈⟦Δ'⟧$.
More specifically, $e' = \Do v$, $v∈⟦τ_1⟧_{ηη'}$ and $μ = ⟦τ_2⟧_{ηη'}$.
We have $\Handle e\,h →^* \Handle K[\Do v]\,h → e_h\subst{v}{x}\subst{v_c}{r}$,
where $v_c=λz.\,\Handle K[z]\,h$.
To show $v_c ∈ ⟦τ_2 →_ρ τ_r⟧_{ηη'}$, take any $u ∈ ⟦τ_2⟧_{ηη'}$ and show
$\Handle K[u]\,h ∈ \E⟦τ_r/ρ⟧_{ηη'} = \E⟦τ_r/ρ⟧_η$.
Which holds by induction hypothesis.
Therefore, $e_h\subst{v}{x}\subst{v_c}{r}$ is in $\E⟦τ_r/ρ⟧_η$ and so is $\Handle e\;h$.

If $n>0$, then
$\Handle K[e']\,h ∈ \S⟦τ_r/ρ⟧_η$,
since $n-1\Free(\Handle K\,h)$, $(e',n-1,μ) ∈ ⟦ρ⟧_η$,
% gdyby przestrzeń efektów pozwalała na niezerowe indeksy to by się to psuło!
and $∀u∈μ.\,\Handle K[u]\,h ∈ \E⟦τ_r/ρ⟧_η$.
Again, the claim follows by antireduction.

\end{proof}

% Lemma: type substitution preserves well-kinding?

% Ahmed LR: "compositionality"
\begin{lemma} \label{subst-comp}
	If $Δ$ and $Δ'$ disjoint, $Δ ⊢ δ ∷ Δ'$, and $Δ,Δ' ⊢ τ ∷ κ$, $η∈⟦Δ⟧$, and
	$η'$ extends $η$ by mappings $α↦⟦δ(α)⟧_η$,
	then $⟦τ⟧_{η'} = ⟦δ(τ)⟧_η$.
\end{lemma}
\begin{proof}
	By induction on the kinding rules.

	If $τ=α ∈ Δ$, then both sides are equal to $η(α)$.

	If $τ=α ∈ Δ'$, then equality follows from the definition of $η'$.

	If $τ=ι$, then both sides are empty.

	If $τ=∀α∷κ.\,τ'$, then
	$⟦τ⟧_{η'} = \bigcap \{⟦τ'⟧_{η'[α↦μ]} │ μ∈⟦κ⟧\}$
	and
	$⟦δ(τ)⟧_{η} = \bigcap \{⟦δ(τ')⟧_{η[α↦μ]} │ μ∈⟦κ⟧\}$,
	which are equal by the inductive hypothesis
	(taking $Δ, α∷κ$ as $Δ$ in the statement).

	If $τ=Δ''.\,τ_1 \Rightarrow τ_2$,
	then $⟦τ⟧_{η'} = \{(e,0,⟦τ_2⟧_{η'η''}) │ η'' ∈ ⟦Δ''⟧ ∧ v ∈ ⟦τ_1⟧_{η'η''} \}$
	and $⟦δ(τ)⟧_{η} = \{(e,0,⟦δ(τ_2)⟧_{ηη''}) │ η'' ∈ ⟦Δ''⟧ ∧ v ∈ ⟦δ(τ_1)⟧_{ηη''} \}$,
	which are equal q`by induction (taking $Δ, Δ''$ as $Δ$ in the statement).

	If $τ=ε·ρ$,
	then $⟦τ⟧_{η'} = ⟦ε⟧_{η'} ∪ \{(e,n+1,μ) │ (e,n,μ) ∈ ⟦ρ⟧_{η'} \}$
	and $⟦δ(τ)⟧_{η} = ⟦δ(ε)⟧_{η} ∪ \{(e,n+1,μ) │ (e,n,μ) ∈ ⟦δ(ρ)⟧_{η} \}$,
	which are equal by the inductive hypothesis.
\end{proof}

\begin{lemma}[Do compatibility]
	$$
	\inferrule
		{Δ; ⊨ v : δ(τ_1) ╱ ι \\ Δ ⊢ δ ∷ Δ' \\ Δ ⊢ Δ'.\, τ_1 \Rightarrow τ_2 ∷ \kE}
		{Δ; ⊨ \Do v : δ(τ_2) ╱ (Δ'.\, τ_1 \Rightarrow τ_2)}
	$$
\end{lemma}
\begin{proof}
Assume $Δ ⊢ δ ∷ Δ'$, $Δ⊢Δ'.\,τ_1→τ_2 ∷ \kE$.
Take any $η ∈ ⟦Δ⟧$.
Assume $v ∈ \E⟦δ(τ_1)/ι⟧_η$.
Clearly, $v ∈ ⟦δ(τ_1)⟧_η$.
We want to show $\Do v ∈ \E⟦δ(τ_2)/(Δ'.\,τ_1 \Rightarrow τ_2)⟧_η$.

By \cref{stuck-inclusion} it suffices to show
$\Do v ∈ \S⟦δ(τ_2)/(Δ'.\,τ_1 \Rightarrow τ_2)⟧_η$.
By taking the empty context in the definition of $\S$ and \cref{value-inclusion},
it suffices to show $(\Do v,0,⟦δ(τ_2)⟧_η) ∈ ⟦Δ'.\,τ_1 \Rightarrow τ_2⟧_η$.
By the interpretation of polymorphic effects, it
would be enough to show $(\Do v,0,⟦δ(τ_2)⟧_η) ∈ ⟦τ_1 \Rightarrow τ_2⟧_{η'}$,
where $η'$ extends $η$ by mappings $α↦⟦δ(α)⟧_η$ for all $α∈Δ'$.
In other words, we need
$v ∈ ⟦τ_1⟧_{η'}$ and $⟦δ(τ_2)⟧_η = ⟦τ_2⟧_{η'}$,
which follows immediately from \cref{subst-comp}.
\end{proof}

\begin{lemma}[Subtyping compatibility]
	$$
	\inferrule
		{Δ ⊢ τ_1 <: τ_2 \\ Δ ⊢ ρ_1 <: ρ_2 \\ Δ; ⊨ e : τ_1 ╱ ρ_1}
		{Δ; ⊨ e : τ_2 ╱ ρ_2}
	$$
\end{lemma}
\begin{proof}
By induction on subtyping rules we will show that
if $Δ ⊢ σ_1 <: σ_2$, then $⟦σ_1⟧_η ⊆ ⟦σ_2⟧_η$ for all $η ∈ ⟦Δ⟧$.
Then, by \cref{mono}, from $Δ ⊢ τ_1 <: τ_2$, $Δ ⊢ ρ_1 <: ρ_2$ and $Δ; Γ ⊨ e : τ_1╱ρ_1$
we will be able to conclude
$Δ; Γ ⊨ e : τ_2╱ρ_2$.

For the case of the reflexivity rule,
we obviously have $⟦σ⟧_η ⊆ ⟦σ⟧_η$.

For the case of the function type rule,
assume $⟦τ_2^1⟧_η ⊆ ⟦τ_1^1⟧_η$, $⟦ρ_1⟧_η ⊆ ⟦ρ_2⟧_η$, and $⟦τ_1^2⟧_η ⊆ ⟦τ_2^2⟧_η$.
We want to show $⟦τ_1^1 →_{ρ_1} τ_1^2⟧_η ⊆ ⟦τ_2^1 →_{ρ_2} τ_2^2⟧_η$.
So take any $(λx.\,e)$ in the former and any $v ∈ ⟦τ_2^1⟧$.
Since $v ∈ ⟦τ_1^1⟧_η$, we have $e\subst{v}{x} ∈ \E⟦τ_1^2/ρ_1⟧_η$.
By \cref{mono} we obtain $e\subst{v}{x} ∈ \E⟦τ_2^2/ρ_2⟧_η$ as desired.

For the case of the universal quantifier rule,
assume $⟦τ_1⟧_{η[α↦μ]} ⊆ ⟦τ_2⟧_{η[α↦μ]}$ for all $μ ∈ ⟦κ⟧$.
The claim holds since
\begin{align*}
⟦∀α∷κ.\,τ_1⟧_η
&= \{ v \mid ∀μ∈⟦κ⟧.\, v ∈ ⟦τ_1⟧_{[α↦μ]η} \} \\
&⊆ \{ v \mid ∀μ∈⟦κ⟧.\, v ∈ ⟦τ_2⟧_{[α↦μ]η} \}
=
⟦∀α∷κ.\,τ_2⟧_η.
\end{align*}

The case of the empty row rule holds trivially, since $⟦ι⟧_η = ∅$.

For the case of the row extension rule, assume $⟦ρ_1⟧_η ⊆ ⟦ρ_2⟧_η$.
We clearly have
$$\{(e,n+1,μ) │ (e,n,μ) ∈ ⟦ρ_1⟧_η\} ⊆ \{(e,n+1,μ) │ (e,n,μ) ∈ ⟦ρ_2⟧_η\},$$
so $⟦ε·ρ_1⟧_η ⊆ ⟦ε·ρ_2⟧_η$ as well.

\end{proof}

\begin{lemma}[Polymorphism introduction compatibility]
	$$
	\inferrule
		{Δ,α∷κ; ⊨ e : τ ╱ ι}
		{Δ; ⊨ e : ∀α∷κ.\,τ ╱ ι}
	$$
\end{lemma}
\begin{proof}
Assume $Δ,α∷κ; ⊨ e : τ ╱ ι$.
Take $η∈⟦Δ⟧$.
We know $e$ evaluates to a value in $⟦τ⟧_{η[α↦μ]}$ for
any $μ∈⟦κ⟧$.
Therefore this value is in $⟦∀α∷κ.\,τ⟧_η$,
and by antireduction $e ∈ \E⟦∀α∷κ.\,τ/ι⟧_η$.
\end{proof}

\begin{lemma}[Polymorphism elimination compatibility]
	$$
	\inferrule
		{Δ ⊢ σ ∷ κ \\ Δ; ⊨ e : ∀α∷κ.\, τ ╱ ρ}
		{Δ; ⊨ e : τ\subst{σ}{α} ╱ ρ}
	$$
\end{lemma}
\begin{proof}
Assume $Δ ⊢ σ ∷ κ$ and $Δ; ⊨ e : ∀α∷κ.\,τ╱ρ$.
By \cref{subst-comp} we have $⟦τ\subst{σ}{α}⟧_η = ⟦τ⟧_{η[α↦⟦σ⟧_η]}$,
which is a superset of $⟦∀α∷κ.\,τ⟧$.
So we have $Δ; ⊨ e : τ\subst{σ}{α} ╱ ρ$ by \cref{mono}.
\end{proof}

\begin{theorem}[Termination of evaluation]
	If $⊢ e : τ ╱ ι$, then evalution of $e$ terminates.
\end{theorem}
\begin{proof}
Take a derivation of $⊢ e : τ ╱ ι$.
After replacing each typing rule by the corresponding compatibility lemma,
we obtain $⊨ e : τ ╱ ι$, so $e ∈ \E⟦τ/ι⟧$.
Therefore $e$ has to terminate to a value, since
$⟦ι⟧$ is empty and hence $\S⟦τ/ι⟧$ is empty.
\end{proof}

This is also an alternative proof of soundness (``well-typed programs do not go wrong''), which was already shown by
the technique of progress and preservation in \cite{fscd19}.

\section{Extensions}

The calculus considered so far is fairly simple.
We will explain why the method we have shown scales to different calculi.

A simple extension we can make is analogous to multiple prompts for delimited continuations:
annotate each operation ($\keyword{do}_l \;v$), handler
($\keyword{handle}_l\;e\;h$), and lift ($\Lift{e}_l$) with a label $l$ from a
fixed set of labels $\mathcal{L}$.
As a result, we obtain sets of completely independent effectful operators for each label.
We would also need a notion of freeness per label,
which would naturally be reflected in the type system as having different rows of effect per label.\footnote{
	Alternatively: one row in which effects are tagged with labels and consecutive effects with different tags can be swapped using subsumption rules.
}
The numbers representing freeness in the logical relations would also have to be indexed by labels,
i.e.\ changed into
functions from labels to natural numbers.
The proofs do not change significantly.

A common feature in languages with algebraic effects is grouping
multiple operations into one effect.
We argue that this is not a significant change and is not deeply related
to effects,
as it could be simulated if the language featured GADTs with pattern matching.
If we want an effect $ε$ with operations $Δ'_i.\,op_i : τ_i \Rightarrow σ_i$,
we can define a GADT $e\;α$ with constructors $op_i : ∀Δ'_i.\,τ_i → e\;σ_i$
and set $ε = α::\kT.\, e\;α \Rightarrow α$.
A handler for this effect would have to pattern match on a value of type $e\;α$
to refine the type variable $α$.
% TODO: GADT width subtyping?

Nevertheless, we can sketch how a direct extension to support grouping operations into one effect would look like.
We fix a set of operation names ranged over by $op$.
Effects are now sets of distinct operation names together with their types.
$$
	\inferrule
	{\textrm{for each }i ∈ \{1…n\} \\ Δ,Δ'_i ⊢ τ_i ∷ \kT \\ Δ,Δ'_i ⊢ σ_i ∷ \kT}
	{Δ ⊢ \{op_i : Δ'_i.\, τ_i \Rightarrow σ_i │ i ∈ \{1…n\}\} ∷ \kE }
$$
Handlers would take on the form $\Handle e\;\{op_1\;x\;r.\,e_1; …; op_n\;x\;r.\,e_n; x.\,e_r \}$
and we would have $op\;v$ instead of $\Do v$.
The typing rule for operations would also have to ``guess'' the other operations in an effect
(in the original rule it already guesses the types in the operation and a type-level substitution).
The rule for $\Handle\!\!$ would naturally need a suitable premise for each operation clause.

When we have both features and identify labels with effects (groups of operations),%
\footnote{
	Technically, effects are formed in type contexts. Therefore,
	we also forbid effects to have free type variables, so that they can be formed
	at the ``top level'', where the set of labels is also fixed.
	This is similar to effects in \cite{hwc}.
	We also note that with this restriction, effect rows could be represented
	simply as mappings from labels to natural numbers.
}
we arrive at a calculus very similar to the one in \cite{hwc}.
A crucial difference that remains is our effect types are still structural.
In particular, effects cannot be recursive.
In the work cited it is conjectured that a restriction on recursive occurrences of effects would
make their calculus terminating.
Therefore, we claim that our method settles this conjecture positively.
We refrain from expounding the details, as the interpretations
of effects would be essentially the same as in the cited work (but unary)
and the definitions of $\E$ and $\S$ would be inductive as in this work.
As is usual with logical relations, once the definitions are reasonable,
the proofs are straightforward, so we omit them too.

We consider the type system in \cite{hwc} to be powerful enough to represent
a large portion of languages with algebraic effects.
In particular, it features polymorphism and allows duplicate instances of effects,
both of which are sometimes omitted in the literature.

\section{Related work}

One thing that was not mentioned about \cite{hwc}, where the interpretation of effects
and the relation on control-stuck terms comes from,
is that their relation is {\em biorthogonal}
– there is an additional relation on (full program) evaluation contexts.
Their relation on expressions is defined extensionally:
an expression is ``good'' if it behaves ``well'' in all ``good'' contexts.
In our work, noticing that this is unnecessary led to simpler definitions and proofs.
More importantly, our intensional direct-style definition of good expressions leads to
better intuition, in particular, allows us to consider reduction traces.
However, opting for direct-style logical relations would possibly not simplify the work
in \cite{hwc} dramatically,
as the relations are binary.

A work which is eerily similar to ours, albeit in a very different setting, is \cite{marriage}.
To reason about program equivalence,
they define a relation $\E$ on expressions
and a relation $\S$ on stuck terms
by solving a recursive equation.
But the terms are ``stuck'' for a different reason:
they are applications to ``external'' functions about which we in a sense do not know enough,
a situation reminiscent of normal form bisimulations.
Furthermore, they interpret the recursive definition {\em coinductively},
i.e. as the greatest fixed point.

In our setting, a coinductive definition would theoretically allow
expressions in $\E⟦τ/ρ⟧_η$ to have reduction traces that are ill-founded
i.e.\ contain infinite paths.
%i.e. some choices of operation return values would lead to nontermination.
The compatibility lemmas can be proven using strong coinduction\footnote{
	Strong coinduction is the principle that $X ⊆ F(X ∪ νF)$ implies $X ⊆ νF$,
	where $F$ is an operator on a complete lattice and $νF$ is its greatest fixed point.
}
	at the type in the rule conclusion,
except the one for $\Handle\!\!$.
The problem is that we substitute a resumption
(which contains a handler) for $r$,
but to show that it is in $⟦τ_2 →_ρ τ_r⟧_η$
we would need exactly the claim we are trying to prove in the first place.
The failure of $\Handle\!\!$ compatibility should not be very surprising,
as this is the rule responsible for eliminating effects –
in particular it should be able to take
any expression in $\E⟦τ/τ_1 \Rightarrow τ_2⟧_η$ and show that wrapping it in a
suitable handler produces an expression in $\E⟦τ/ι⟧_η$, which cannot get
stuck and simply evaluates to a value of type $τ$.
This seems impossible,
as theoretically the expression's reduction trace could have no
leaves, so by definition of $\E$ the type $τ$ could be arbitrary.

We have found only sketches of normalization proofs for algebraic effects in the literature,
so we cannot be sure of how they work.
Such a mention appears in \cite{hia}.
We speculate that their proof might have
a combinatorial flavor, rather than use only logical relations.
This is because multiple such arguments appear in a paper they cite \cite{exrw} and
because of the mention that the handler is reapplied
on a strict subterm of the original computation during each handling reduction.
This is possible thanks to an unusual calculus,
where the operation carries a continuation alongside the argument
and there is a reduction that moves frames surrounding the expression to the continuation until
the operation is directly inside the handler.
It is feasible that such an argument could also work in our semantics by using
a suitable term size function.

A calculus with effect handlers appears and its termination proof is mentioned in \cite{expressive}.
As it is even terser than the previous one, we cannot give any comment.

A mechanized proof of strong normalization for the calculus defined in \cite{expressive}
is reported in \cite{sol}.
A tool for proof search is used and the proof is not spelled out in detail,
but it also relies on combinatorial arguments and strikes us as more elaborate than ours.

We did not consider recursive effects, as they are known to introduce nontermination \cite{hwc}.
Likely, we could allow effects where all recursive occurrences are positive
– but note that
the left side of an effect arrow ($\Rightarrow$) is positive and the right is negative,
as opposed to function arrows ($→$).
Needless to say, a least fixed point construction would have to be used in the interpretation of effects.
This is discussed from the perspective of a translation into a System F-like calculus in
\cite{xie2020effect}.

The possibility of defining logical relations by solving fixed-point equations
has been known for a long time.
This is most natural when the language considered has features directly
concerning fixed points, such as (co)inductive types \cite{altenkirch, operfl}.
In different settings, this technique appears rare, and our work is one example.

% would the proof even work with biorthogonality?


\printbibliography

\end{document}
